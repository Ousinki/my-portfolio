\documentclass{ctexart}
\usepackage{graphicx} % Required for inserting images
\usepackage{geometry} % 设置页边距
\usepackage{lipsum} % 生成虚拟文本
\usepackage{fancyhdr} % 自定义页眉页脚
\usepackage{lastpage} 
\usepackage{comment} % 添加多行注释
\usepackage{listings} % 添加代码高亮
\usepackage{xcolor}
% \usepackage{fontspec}
% \usepackage{xeCJK}
\usepackage{minted}


\lstset{
  backgroundcolor=\color{black!5},   % 设置背景色
  basicstyle=\ttfamily\small,        % 设置代码字体
  breaklines=true,                   % 自动换行
  commentstyle=\color{green!80!black}, % 设置注释颜色
  keywordstyle=\color{blue},         % 设置关键词颜色
  stringstyle=\color{purple},        % 设置字符串颜色
  showstringspaces=false,            % 不显示字符串中的空格
  numbers=left,                      % 在左侧显示行号
  numberstyle=\tiny\color{gray},     % 行号样式
  frame=single,                      % 添加边框
  % --- 下面是新加的关键一行 ---
  literate={"}{{"}}1 {'}{{'}}1      % 将直角引号正确显示
}



\title{LaTeX教程}
\author{Green Sleeves}
%\date{July 2025}
\date{\today}

\geometry{left=2.5cm,right=2.5cm,top=2cm,bottom=2cm}
\setlength{\headheight}{12.64723pt}
%\pagestyle{empty} % 页眉页脚没东西,但页脚中间的1是默认有的
\pagestyle{fancy}
\fancyhf{}
\lhead{myname\_Green}
\rhead{\LaTeX{}教程}
\cfoot{第 \thepage 页(共 \pageref*{LastPage} 页)}
\lstset{basicstyle=\ttfamily}
% 设置中文字体,确保名称与字体册中的完全一致
% \setCJKmainfont{方正屏显雅宋简体}



\begin{document}

% 封面页(不要页码、不计入总页数)
\pagenumbering{gobble}  % 不显示页码
\maketitle % 显示信息栏的三行内容:title,author,date
%\thispagestyle{empty} % 设置该页的页眉页脚样式为 “空”

\newpage
\pagenumbering{arabic}  % 恢复页码显示
\setcounter{page}{1}    % 从 1 开始计数
\tableofcontents % 根据文件大纲生成目录

% 正文开始
\newpage % 让正文页开启新的一页

\section{章节}
\subsection{各级标题}
\lstinline|\section{章}| 

\hspace{2em}\lstinline|\subsection{节}| 

\hspace{4em}\lstinline|\subsubsection{小节}| 

\hspace{6em}\lstinline|\paragraph{段落标题}| 

\hspace{8em}\lstinline|\subparagraph{子段落标题}| 

\vspace{5mm} % 增加 5 毫米的垂直空白

“Sub” 是一个非常常见的英文前缀 (prefix)。它的基本意思是:在...之下 (under, below);次要的、副的 (secondary, vice-);部分的 (part of)。\lstinline|\section| 和 \lstinline|\subsection|的这个 sub 就是用来表示层级关系的。

\paragraph{段落标题}与正文内容在同一行。

\subparagraph{子段落标题}与正文内容在同一行,并有首行缩进。

\subsection{代码}
在 LaTex 中,我们使用 \lstinline|\lstinline| 来输出行内代码。lst 是 listing (列表、清单) 的缩写,在 listings 这个宏包的语境里,它特指“代码清单 (code listing)”;inline 是一个排版和网页设计里的常用词,意思是“行内”。

下面的代码则展示了如何输出代码块:
\begin{lstlisting}[language=Python]
print("Hi, lstlisting!")
\end{lstlisting}

\begin{minted}{javascript}
    var name = "minted";
    console.log(name);
\end{minted}

% 添加了行号(linenos)和边框(frame=lines)
\begin{minted}[linenos, frame=lines]{javascript}
    var name = "minted";
    console.log("Hello, " + name);
\end{minted}

lstlisting 环境 minted 环境 都是用于输出代码块的,但是有以下区别:
\begin{itemize}
    \item 语法高亮:minted 环境支持更丰富的语法高亮,而 listings 环境仅支持有限的语言。
    \item 代码片段:minted 环境更适合展示完整的代码片段,而 listings 环境更适合展示单行的代码。
\end{itemize}

lstlisting 内置于 TeX,它完全自给自足,不依赖任何外部件,只要有 LaTeX 就能跑。虽然功能基础,但非常稳定可靠。

minted 需要依赖外部 Python 和 Pygments 库协同工作,配置起来更复杂,但它能提供的性能(高亮效果)远超前者。它需要: 1. 安装 Python 和 Pygments (pip install) 2. 编译器开启 -shell-escape 选项

minted 环境同样提供了专门用来处理行内代码的命令。在 Python 中,我们使用 \mintinline{python}{print("Hello")} 函数来输出文本。对于 JavaScript,可以这样写:调用 \mintinline{javascript}{console.log("Hi")} 函数。C++ 的 main 函数是这样定义的:\mintinline{cpp}|int main() { return 0; }|。


\subsection{虚拟文本}
\lstinline|\lipsum| 用于自动生成一段被称为 “Lorem Ipsum” 的虚拟文本或占位文本 (placeholder text),一共有150段预定义的虚拟文本。它是一段从古罗马作家西塞罗的文章《论善与恶的极致》中提取出来的拉丁文,但经过了打乱和修改,所以内容是完全无意义的。正因为它无意义,所以读者不会被内容吸引,从而能更好地审视排版和设计。

\lipsum[1] % 生成第1段虚拟文本

\subsection{手书文本}
这是第一段。

这是第二段。(源代码中空一行,表示换行。)

\subsubsection{格式设置}
\textbf{加粗} \textit{italic斜体} \underline{下划线} \underline{\textbf{嵌套}}
% 在overleaf里仍然可以使用快捷键ctrl+B等等


\subsubsection{列表}
\begin{enumerate} % 有序列表
    \item Enumerate 是一个英文动词,意思是“列举”、“枚举”、“计数”。它的词根和 "number" (数字) 有关。
    \item Item 是一个英文名词,意思是“项目”、“条目”。
    \item c
    \item d
\end{enumerate}

\begin{itemize} % 无序列表
    \item Itemize 是一个英文动词,源自名词 item (项目),意思就是“分条列出”。
    \item b
    \item c
    \item d
\end{itemize}


\begin{figure}[h] % [h] 将图片放置在here
    \centering % 图片居中
%    \includegraphics[width=0.1\linewidth]{image.png} 
    \caption{Enter Caption}
    \label{fig:enter-label}
\end{figure}
% linewidth 当前容器中的文字宽度(局部)

\newpage

\begin{figure} [b] % [b] 将图片放置在bottom
    \centering
%    \includegraphics[width=0.1\textwidth]{image.png}
    \caption{Enter Caption}
    \label{fig:enter-label}
\end{figure}
% textwidth 整个页面的文字区宽度(全局)



\begin{comment}
这是一个多行注释,
使用的是 \usepackage{comment} 
\end{comment}

% 这也是多行注释,
% 选中文本后,按cmd+/







\end{document}





